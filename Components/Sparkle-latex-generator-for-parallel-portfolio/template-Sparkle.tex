\documentclass[british]{article}
\usepackage[T1]{fontenc}
\usepackage[latin9]{inputenc}
\usepackage{geometry}
\geometry{verbose,tmargin=3.5cm,bmargin=3.5cm,lmargin=3cm,rmargin=3cm}
\usepackage{array}
\usepackage{amstext}
\usepackage{graphicx}
\usepackage{color}
\usepackage{caption}

\newcommand{\tabincell}[2]{\begin{tabular}{@{}#1@{}}#2\end{tabular}}

\newcommand{\mytablefontsize}{7pt}
\newcommand{\mytablebaselineskip}{0.7}
\newcommand{\mytabcolsep}{3pt}

\newcommand{\medianInterval}[1]{}
@@customCommands@@

\makeatletter

\newif\ifdes
@@testBool@@

%%%%%%%%%%%%%%%%%%%%%%%%%%%%%% LyX specific LaTeX commands.
%% Because html converters don't know tabularnewline
\providecommand{\tabularnewline}{\\}

%%%%%%%%%%%%%%%%%%%%%%%%%%%%%% User specified LaTeX commands.

\title{ Parallel portfolio report}
\author{ @@sparkle@@ }

\makeatother

\usepackage{babel}
\begin{document}
\maketitle %


\section{Introduction}
\label{sec:Introduction}

@@sparkle@@ \cite{Hoos15} is a multi-agent problem-solving platform based on Programming by Optimisation (PbO) \cite{Hoos12}, and would provide a number of effective algorithm optimisation techniques (such as automated algorithm configuration, portfolio-based algorithm selection, etc.) to accelerate the existing solvers.

This experimental report is automatically generated by @@sparkle@@. This report presents experimental results of @@sparkle@@ parallel portfolio containing @@numSolvers@@ solver(s).

\section{Experimental Preliminaries}
\label{sec:Experimental_Preliminaries}

This section presents the experimental preliminaries, including the list of solvers in the portfolio, the list of instance sets and information about the experimental setup.

\subsection{Solvers}
\label{sec:Solvers}
There are @@numSolvers@@ solver(s) included in @@sparkle@@, as listed below.

\begin{enumerate} 
@@solverList@@
\end{enumerate}

\subsection{Instance Set(s)}
\label{sec:Instance_Sets}
There are @@numInstanceClasses@@ instance set(s) included in @@sparkle@@, as listed below.

\begin{enumerate}
@@instanceClassList@@
\end{enumerate}

\subsection{Experimental Setup}
\label{sec:Experimental_Setup}

The experimental setup is described below.

\ifdes \textbf{Performance computation:} @@sparkle@@ runs the portfolio one time on each instance. The cutoff time for the computation run is set to @@cutoffTime@@ seconds. The outcome of the computation is listed below. The scores of the outcomes are calculated according to PAR10, this means that for each instance the solver which solved the instance is scored its runtime and the remaining solvers are scored the runtime times ten. If however the porfolio reaches the cutofftime, which means that no solvers solved the instance, all solvers are scored the cutofftime times ten.
\else \textbf{Optimization results:} @@sparkle@@ runs the portfolio one time on each instance. The cutoff time for the computation run is set to @@cutoffTime@@ seconds. The outcome of the computation is listed below. \fi

\begin{enumerate}
@@solversWithSolution@@
\end{enumerate}

\subsection{Scatter Plot Analysis}
\label{sec:Scatter_Plot_Analysis}

Figure~\ref{fig:sparkle_vs_sbs} shows the empirical comparison between the actual parallel portfolio in @@sparkle@@ and the single best solver (\emph{SBS}).

% \begin{figure}[htbp]
% \noindent \begin{centering}
%     @@figure-parallel-portfolio-sparkle-vs-vbs@@
% \par\end{centering}
% 
% \end{figure}

\begin{figure}[htbp]
\noindent \begin{centering}
    @@figure-parallel-portfolio-sparkle-vs-sbs@@
\par\end{centering}

\caption{Empirical comparison between the actual parallel portfolio in @@sparkle@@ and the \emph{SBS}.}\label{fig:sparkle_vs_sbs}
\end{figure}

\bibliographystyle{plain}
\bibliography{Sparkle_Report}

\end{document}
