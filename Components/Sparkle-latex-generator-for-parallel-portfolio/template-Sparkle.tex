\documentclass[british]{article}
\usepackage[T1]{fontenc}
\usepackage[latin9]{inputenc}
\usepackage{geometry}
\geometry{verbose,tmargin=3.5cm,bmargin=3.5cm,lmargin=3cm,rmargin=3cm}
\usepackage{array}
\usepackage{amstext}
\usepackage{graphicx}
\usepackage{color}
\usepackage{caption}

\newcommand{\tabincell}[2]{\begin{tabular}{@{}#1@{}}#2\end{tabular}}

\newcommand{\mytablefontsize}{7pt}
\newcommand{\mytablebaselineskip}{0.7}
\newcommand{\mytabcolsep}{3pt}

\newcommand{\medianInterval}[1]{}
@@customCommands@@

\makeatletter

%%%%%%%%%%%%%%%%%%%%%%%%%%%%%% LyX specific LaTeX commands.
%% Because html converters don't know tabularnewline
\providecommand{\tabularnewline}{\\}

%%%%%%%%%%%%%%%%%%%%%%%%%%%%%% User specified LaTeX commands.

\title{Experimental Reports for @@sparkle@@ }
\author{ @@sparkle@@ }

\makeatother

\usepackage{babel}
\begin{document}
\maketitle %


\section{Introduction}
\label{sec:Introduction}

@@sparkle@@ \cite{Hoos15} is a multi-agent problem-solving platform based on Programming by Optimisation (PbO) \cite{Hoos12}, and would provide a number of effective algorithm optimisation techniques (such as automated algorithm configuration, portfolio-based algorithm selection, etc.) to accelerate the existing solvers.

This experimental report is automatically generated by @@sparkle@@. This report presents experimental results of @@sparkle@@ parallel portfolio ran on the instance set @@testInstanceClass@@.

\section{Experimental Preliminaries}
\label{sec:Experimental_Preliminaries}

This section presents the experimental preliminaries, including the list of solvers in the portfolio(s), the list of instance sets and information about the experimental setup.

\subsection{Solvers}
\label{sec:Solvers}
There are @@numSolvers@@ solver(s) included in @@sparkle@@, as listed below.

\begin{enumerate} 
@@solverList@@
\end{enumerate}

\subsection{Instance Set(s)}
\label{sec:Instance_Sets}
There are @@numInstanceClasses@@ instance set(s) included in @@sparkle@@, as listed below.

\begin{enumerate}
@@instanceClassList@@
\end{enumerate}

\subsection{Experimental Setup}
\label{sec:Experimental_Setup}

The experimental setup is described below.

\textbf{Performance computation:} @@sparkle@@ runs the portfolio one time on each instance. The cutoff time for the computation run is set to @@cutoffTime@@ seconds.

\bibliographystyle{plain}
\bibliography{Sparkle_Report}

\end{document}
