\documentclass[british]{article}
\usepackage[T1]{fontenc}
\usepackage[latin9]{inputenc}
\usepackage{geometry}
\geometry{verbose,tmargin=3.5cm,bmargin=3.5cm,lmargin=3cm,rmargin=3cm}
\usepackage{array}
\usepackage{amstext}
\usepackage{graphicx}
\usepackage{color}
\usepackage{caption}

\newcommand{\tabincell}[2]{\begin{tabular}{@{}#1@{}}#2\end{tabular}}

\newcommand{\mytablefontsize}{7pt}
\newcommand{\mytablebaselineskip}{0.7}
\newcommand{\mytabcolsep}{3pt}

\newcommand{\medianInterval}[1]{}
@@customCommands@@

\makeatletter

%%%%%%%%%%%%%%%%%%%%%%%%%%%%%% LyX specific LaTeX commands.
%% Because html converters don't know tabularnewline
\providecommand{\tabularnewline}{\\}

%%%%%%%%%%%%%%%%%%%%%%%%%%%%%% User specified LaTeX commands.

\title{Experimental Reports for @@sparkle@@ }
\author{ @@sparkle@@ }

\makeatother

\usepackage{babel}
\begin{document}
\maketitle %


\section{Introduction}
\label{sec:Introduction}

@@sparkle@@ \cite{Hoos15} is a multi-agent problem-solving platform based on Programming by Optimisation (PbO) \cite{Hoos12}, and would provide a number of effective algorithm optimisation techniques (such as automated algorithm configuration, portfolio-based algorithm selection, etc.) to accelerate the existing solvers.

This experimental report is automatically generated by @@sparkle@@. This report presents experimental results on the scenario of running the @@sparkle@@ portfolio selector to solve the test instances in the instance set @@testInstanceClass@@.


\section{Information about the Test Instance Set}
\label{sec:Information_about_the_Test_Instance_Class}

\begin{itemize}
\item Testing set: @@testInstanceClass@@, consisting of @@numInstanceInTestInstanceClass@@ instances
\end{itemize}


\section{Information about the Current @@sparkle@@ Platform}
\label{sec:Information_about_Current_Sparkle}

This section presents the experimental preliminaries, including the list of solvers, the list of feature extractors, the list of instance sets, information about the experimental setup and information about how to construct a portfolio-based algorithm selector in @@sparkle@@.

\subsection{Solvers}
\label{sec:Solvers}
There are @@numSolvers@@ solver(s) included in @@sparkle@@, as listed below.

\begin{enumerate} 
@@solverList@@
\end{enumerate}

\subsection{Feature Extractors}
\label{sec:Feature_Extractors}
There are @@numFeatureExtractors@@ feature extractor(s) included in @@sparkle@@, as listed below.

\begin{enumerate}
@@featureExtractorList@@
\end{enumerate}

\subsection{Train Instance Classes}
\label{sec:Train_Instance_Classes}
There are @@numInstanceClasses@@ instance class(es) included in @@sparkle@@, as listed below.

\begin{enumerate}
@@instanceClassList@@
\end{enumerate}

\subsection{Experimental Setup for Training Phase}
\label{sec:Experimental_Setup_for_Training_Phase}

The experimental setup for the training phase is described below.

\textbf{Feature computation:} @@sparkle@@ uses all the feature extractors presented above to compute a feature vector for each training instance. Every feature extractor computes a feature vector for each training instance. The final feature vector is then the combination of all computed feature vectors. The cutoff time for feature vector computation on each training instance is set to @@featureComputationCutoffTime@@ seconds. This time is shared equally between the feature extractors.

\textbf{Performance computation:} @@sparkle@@ runs each solver one time on each training instance. The cutoff time for each performance computation run is set to @@performanceComputationCutoffTime@@ seconds.

\subsection{Constructing the Portfolio-Based Algorithm Selector}
\label{sec:Portfolio}

@@sparkle@@ saves the results of feature extraction and performance computation described above. This data is then utilized by @@sparkle@@ to run \emph{AutoFolio} \cite{LinEtAl15} to automatically construct a portfolio-based algorithm selector for @@sparkle@@.


\section{Experimental Results}
\label{sec:Experimental_Results}

In this section, the PAR10 results for the current portfolio selector in @@sparkle@@ on solving the test instance set @@testInstanceClass@@ is reported.

\begin{itemize}
\item \textbf{Actual Portfolio Selector in @@sparkle@@}, PAR10: @@testActualPAR10@@
\end{itemize}


\subsection{PAR10 Ranking List}
\label{sec:PAR10_Ranking}

The ranking list with regards to the penalised average runtime (PAR10) for solvers is given as follows.

\begin{enumerate}
@@PAR10RankingList@@
\end{enumerate}

Also, PAR10 for the Virtual Best Solver \emph{VBS}, i.e., the perfect portfolio selector, and the actual portfolio selector in @@sparkle@@ is given as follows.

\begin{itemize}
\item \textbf{\emph{VBS}}, PAR10: @@VBSPAR10@@
\item \textbf{Actual Portfolio Selector in @@sparkle@@}, PAR10: @@actualPAR10@@
\end{itemize}

\subsection{Marginal Contribution Ranking List}
\label{sec:Marginal_Contribution_Ranking}

@@sparkle@@ uses the concept of marginal contribution \cite{XuEtAl12} to measure each solver's contribution to the \textbf{\emph{VBS}} and to the \textbf{actual portfolio selector in @@sparkle@@}. In this report, we uses the approach described in the literature \cite{FreEtAl16} to each solver's marginal contribution.

Solver ranking list via marginal contribution \cite{XuEtAl12} for solvers with regards to the \textbf{{\em VBS}} is given as follows.

\begin{enumerate} 
@@solverPerfectRankingList@@
\end{enumerate}

Solver ranking list via marginal contribution \cite{XuEtAl12} for solvers with regards to the \textbf{actual portfolio selector in @@sparkle@@} is given as follows.

\begin{enumerate} 
@@solverActualRankingList@@
\end{enumerate}


\subsection{Scatter Plot Analysis}

The empirical comparison between the actual portfolio selector in @@sparkle@@ and single best solver (\emph{SBS}) is presented in Figure \ref{fig:sparkle_vs_sbs}
The empirical comparison between the actual portfolio selector in @@sparkle@@ and \emph{VBS} is presented in Figure \ref{fig:sparkle_vs_vbs}.

\begin{figure}[htbp]
\noindent \begin{centering}
@@figure-portfolio-selector-sparkle-vs-sbs@@
\par\end{centering}

\caption{Empirical comparison between the actual portfolio selector in @@sparkle@@ and \emph{SBS}.}\label{fig:sparkle_vs_sbs}
\end{figure}

\begin{figure}[htbp]
\noindent \begin{centering}
@@figure-portfolio-selector-sparkle-vs-vbs@@
\par\end{centering}

\caption{Empirical comparison between the actual portfolio selector in @@sparkle@@ and \emph{VBS}.}\label{fig:sparkle_vs_vbs}
\end{figure}


\bibliographystyle{plain}
\bibliography{Sparkle_Report_for_Test}

\end{document}
