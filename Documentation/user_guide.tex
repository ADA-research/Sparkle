\documentclass{article}

\usepackage{todonotes}

\usepackage{listings}
\lstset{%
  basicstyle=\ttfamily,
}

\title{Sparkle user guide}
\author{Koen van der Blom}
\date{\today}

\begin{document}

\maketitle

\section{Settings}

\subsection{Slurm (focused on Grace)}
Slurm settings can be specified in the \texttt{Settings/sparkle\_slurm\_settings.txt} file. Currently these settings are inserted as is in any \texttt{srun} or \texttt{sbatch} calls done by Sparkle. This means that any options exclusive to one or the other currently should not be used (see Subsubsection~\ref{slurm:disallowed}).

\subsubsection{Tested options}
Below a list of tested Slurm options for \texttt{srun} and \texttt{sbatch} is included. Most other options for these commands should also be safe to use (given they are valid), but have not been explicitly tested. Note that any options related to commands other than \texttt{srun} and \texttt{sbatch} should not be used with Sparkle, and should not be included in \texttt{Settings/sparkle\_slurm\_settings.txt}.

\begin{itemize}
  \setlength{\itemsep}{0pt}
  \item[] \texttt{--partition / -p}
  \item[] \texttt{--exclude}
  \item[] \texttt{--nodelist}
\end{itemize}

\subsubsection{Disallowed options}
\label{slurm:disallowed}
The options below are exclusive to \texttt{sbatch} and are thus disallowed:

\begin{itemize}
  \setlength{\itemsep}{0pt}
  \item[] \texttt{--array}
  \item[] \texttt{--clusters}
  \item[] \texttt{--wrap}
\end{itemize}

The options below are exclusive to \texttt{srun} and are thus disallowed:

\begin{itemize}
  \setlength{\itemsep}{0pt}
  \item[] \texttt{--label}
\end{itemize}

\subsubsection{Nested \texttt{srun} calls}
Since a number of Sparkle commands internally call the \texttt{srun} command, and for those commands the provided settings need to match the restrictions of your call to a Sparkle command. Take for instance the following command:

\begin{lstlisting}[breaklines]
srun -N1 -n1 -p graceTST Commands/configure_solver.py -solver Solvers/Yahsp3 -instances-train Instances/Depots_train_few/
\end{lstlisting}

This call restricts itself to the \texttt{graceTST} partition (the \texttt{graceTST} partition only consists of node 22). So if the settings file contains the setting \texttt{--exclude=ethnode22}, all available nodes are excluded, and the command cannot execute any internal \texttt{srun} commands it may have.

Finally, Slurm ignores nested partition settings for \texttt{srun}, but not for \texttt{sbatch}. This means that if you specify the \texttt{graceTST} partition (as above) in your command, but the \texttt{graceADA} partition in the settings file, Slurm will still execute any nested \texttt{srun} commands on the \texttt{graceTST} partition only.

\section{Required packages}

\subsection{Sparkle on Grace}

Grace is the computing cluster of the ADA group\footnote{\texttt{http://ada.liacs.nl/}} at LIACS, Leiden University. Since not all packages required by Sparkle are installed on the system, some have to be installed local to the user.

\subsubsection{\texttt{epstopdf}}
The \texttt{epstopdf} package is required for Sparkle's reporting component to work (e.g. \texttt{generate\_report, generate\_report\_for\_configuration}), it can be installed in your user directory as follows:

\begin{enumerate}
  \item Download \texttt{epstopdf}

  \texttt{wget http://mirrors.ctan.org/support/epstopdf.zip}

  \item Unzip the package

  \texttt{unzip epstopdf.zip}

  \item Rename \texttt{epstopdf.pl} (inside the directory you just unzipped)

  \texttt{mv epstopdf.pl epstopdf}

  \item Add this line to your \texttt{.bashrc} (open with e.g. \texttt{vim \~{}/.bashrc})

  \texttt{export PATH="/<directory>/epstopdf:\$PATH"}

  (replace "\texttt{<directory>}" with the path to the \texttt{epstopdf} directory)
\end{enumerate}

\subsection{Yahsp example}

\begin{enumerate}

  \item Install gmp on Grace

  \texttt{wget https://gmplib.org/download/gmp/gmp-6.1.2.tar.xz}

  \texttt{tar -xf gmp-6.1.2.tar.xz}

  Inside the \texttt{gmp-6.1.2} directory:

  \texttt{./configure}

  \texttt{make}

  \texttt{make check}

  \item Navigate to the \texttt{seq-agl-yahsp3} directory

  \item Add the below after \texttt{-fpermissive} on line 24 of \texttt{cpt-yahsp/CMakeLists.txt}:

  \texttt{ -I /home/blomkvander/lib/gmp-6.1.2/ -L /home/blomkvander/lib/gmp-6.1.2/.libs/}

  (replace \texttt{/home/blomkvander/lib/} with the path where you installed gmp)

  \item Compile yahsp with:

  \texttt{./build}

  \item In \texttt{yahsp/esegui.sh} the line \texttt{\#!/bin/bash} was added to the start of the file to allow Grace nodes to find the 'time' utility.

\end{enumerate}

\end{document}
